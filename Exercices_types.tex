\section{Exercices types}
\subsection{Définition de fonctions complexes spécifiques}
Définir les fonction complexes exponentielle, logarithme, trigonométriques, hypeboliques.

\subsection{Décomposition de fonction complexes}
Savoir décomposer en partie réelle et partie imaginaire toute fonction complexe donnée.

\subsection{Holomorphisme}
Utiliser les équations de Cauchy-Riemann pour déterminer si une fonction complexe est holomorphe. \\
(Série 1 exercice 4, série 2 exercices 1 et 2) \\
\\
\textbf{Méthodologie :}
\begin{itemize}
    \item Décomposer en partie réelle et partie imaginaire la fonction $f(z)$ :
    $$f(z) = u(x,y) + iv(x,y)$$
    \item Déterminer les dérivées partielles selon $x$ et $y$ pour chaque partie :
    \begin{align*}
    u_x(x,y) &= \text{ }? \\
    v_x(x,y) &= \text{ }? \\
    u_y(x,y) &= \text{ }? \\
    v_y(x,y) &= \text{ }?
    \end{align*}
    \item Observer si les dérivées partielles obtenues satisfont les équation de Cauchy-Riemann :
    \begin{align*}
    u_x(x,y) &\overset{?}{=} v_y(x,y) \\
    v_x(x,y) &\overset{?}{=} -u_y(x,y)
    \end{align*}
\end{itemize}
\textbf{Exemple : }(série 1 exercice 4e)\\
Déterminer si $f(z) = \cos(z)$ est holomorphe, si oui donner sa dérivée.
\begin{align*}
    f(z) = \cos(z) &= \frac{e^{iz} + e^{-iz}}{2} \\
    &= \frac{1}{2} \bigg{(} e^{-y+ix} + e^{y-ix} \bigg{)} \\
    &= \frac{1}{2} \bigg{(} e^{-y}(\cos(x) +i\sin(x)) + e^y(cos(x) - i\sin(x)) \bigg{)}
\end{align*}
Donc on a la décomposition $f(z) = u(x,y) + iv(x,y)$ avec :
\begin{align*}
    u(x,y) = \frac{1}{2} \cos(x)(e^{-y} + e^y) \\
    v(x,y) = \frac{1}{2} \sin(x)(e^{-y} - e^y)
\end{align*}
On dérive :
\begin{align*}
    u_x(x,y) &= -\frac{1}{2} \sin(x)(e^{-y} + e^y) \\
    v_x(x,y) &= \frac{1}{2} \cos(x)(e^{-y} - e^y) \\
    u_y(x,y) &= -\frac{1}{2} \cos(x)(e^{-y} - e^y) \\
    v_y(x,y) &= -\frac{1}{2} \sin(x)(e^{-y} + e^y)
\end{align*}
On a $u_x(x,y) = v_y(x,y)$ et $v_x(x,y) = -u_y(x,y)$, les équations de Cauchy-Riemann sont vérifiées. Donc $f(z)$ est holomorphe. \\
On calcule sa dérivée :
\begin{align*}
    f'(z) &= u_x(x,y) + iv_x(x,y) \\
    &= -\frac{1}{2} \sin(x)(e^{-y} + e^y) + i\frac{1}{2} \cos(x)(e^{-y} - e^y) \\
    &= -\frac{1}{2} e^{-y} (\sin(x) - i\cos(x)) -\frac{1}{2} e^{y} (\sin(x) + i\cos(x)) \\
    &= -\frac{1}{2i} e^{-y} (\cos(x) + i\sin(x)) -\frac{1}{2i} e^{y} (-\cos(x) + i\sin(x)) \\
    &= -\frac{1}{2i} e^{-y} (\cos(x) + i\sin(x)) +\frac{1}{2i} e^{y} (\cos(-x) + i\sin(-x)) \\
    &= \frac{1}{2i}(e^{-y+ix} + e^{y-ix}) \\
    &= \frac{1}{2i}(e^{iz} + e^{-iz}) \\
    &= \sin(z)
\end{align*}
\subsection{Détermination du plus grand domaine}
Donner la partie réelle ou imaginaire d’une fonction holomorphe, utiliser les équations de Cauchy-Riemann pour trouver toutes les parties imaginaires ou réelles possibles. \\
(Série 2 exercice 3, série 5 exercice 2) \\
\\
\textbf{Méthodologie :}
\begin{itemize}
    \item Séparer la partie réelle de la partie imaginaire de la fonction
    \item Établir les équations de Cauchy-Riemann et les résoudre 
    \item Intégrer $u_x(x,y)$ ou $v_x(x,y)$ en fonction de $x$, ou $u_y(x,y)$ ou $v_y(x,y)$ en fonction de $y$ pour obtenir $u(x,y)$ ou $v(x,y)$ (ne pas oublier d'ajouter la constante qui dépend de $y$ ou $x$)
    \item Déterminer la constante en dérivant le résultat précédent et en le comparant avec l'équation de Cauchy-Riemann correspondante
    \item Déterminer le plus grand domaine possible en recherchant des restrictions ou des conditions sur les variables $x$ et $y$ qui pourraient rendre la fonction indéfinie ou non holomorphe
\end{itemize}
\textbf{Exemple : }(série 2 exercice 3) \\
Déterminer le plus grand domaine de $\mathbb{C}$ où la fonction $f(z) = \log(1+z^2)$ est holomorphe. \\
\\
$z \mapsto 1+z^2$ est holomorphe sur $\mathbb{C}$ \\
$w \mapsto \log(w)$ est holomorphe sur $\mathbb{C} \setminus \{w \in \mathbb{C} \mid \Im(w) = 0, \Re(w) \leq 0\}$ \\
On pose $w = 1+z^2$ et on détermine l'ensemble des z tels que $\Im(w) = 0, \Re(w) \leq 0$ \\
$w = 1+z^2 = 1 + x^2 -y^2 + i2xy$
\begin{align*}
    \begin{cases}
        \Im(w) = 2xy = 0 \\
        \Re(w) = 1 + x^2 -y^2 \leq 0
    \end{cases}
    \quad
    \Leftrightarrow
    \quad
    \begin{cases}
        x = 0 \\
        y^2 \geq 1
    \end{cases}
\end{align*}
($y=0$ et $x^2+1 \leq 0$ est à exclure) \\
Donc le plus grand domaine est :
\begin{align*}
    D &= \{z = x+iy \mid x = 0, y^2 \geq 1\} \\
    &= \{z \in \mathbb{C} \mid x, \Re(z) = 0, |\Im(z)| \geq 1\}
\end{align*}
\subsection{Intégration complexe}
Calculer des intégrales complexes à l’aide de la définition. \\
(Série 3 exercice 6) \\
\\
\textbf{Méthodologie :}
\begin{itemize}
    \item Choisir une paramétrisation $\gamma(t)$ de $\Gamma$ :
    $$\Gamma = \{\gamma(t) = \text{ } ? \mid t : a \to b\}$$
    \item Calculer l'intégrale avec la définition :
    $$\int_{\Gamma}fdz := \int_a^bf(\gamma(t))\gamma'(t)dt$$
\end{itemize}
\textbf{Exemple : }(série 3 exercice 6) \\
Calculer
$$I = \int_{\Gamma}(z^2+1)dz$$
où $\Gamma$ est le segment joignant $1$ à $1+i$ \\
\\
On choisit une paramétrisation $\gamma(t)$ de $\Gamma$ :
$$\Gamma = \{\gamma(t) = 1+it \mid t : 0 \to 1\}$$
On calcule l'intégrale :
\begin{align*}
    I &= \int_0^1 f(\gamma(t)) \gamma'(t) dt \\
    &= \int_0^1((1+it)^2 +1) i \text{ }dt \\
    &= \int_0^1(2 + 2it -t^2) i \text{ }dt \\
    &= \int_0^1 2i -2t - it^2 dt \\
    &= \bigg{[}2it - t^2 - \frac{it^3}{3}\bigg{]}_0^1 \\
    &= 2i - 1 - \frac{i}{3} \\
    &= -1 + \frac{5i}{3}
\end{align*}
\subsection{Théorème de Cauchy et FIC}
Appliquer le théorème de Cauchy et la formule intégrale de Cauchy pour déterminer l’intégrale complexe d’une fonction holomorphe sur une courbe fermée. \\
(Série 3 exercices 3, 4 et 5, série 4 exercices 1, 2 et 3)\\
\\
\textbf{Méthodologie :}
On veut calculer $I = \int_{\Gamma} f(z)dz$
\begin{itemize}
    \item Si on a $f(z)$ holomorphe sur $int(\Gamma) \cup \Gamma$, alors par le théorème de Cauchy on a $I = 0$
    \item Sinon on essaye d'obtenir $I$ de la forme :
    $$I = \int_{\Gamma} \frac{\widetilde{f}(z)}{(z-z_0)^{n+1}}dz$$
    avec $\widetilde{f}$ holomorphe sur $int(\Gamma) \cup \Gamma$ et $z_0 \in int(\Gamma)$ \\
    Par la FIC (BIS) on obtient alors :
    $$I = 2\pi i \frac{\widetilde{f}^{(n)}(z_0)}{n!}$$
    \item Mais si $z_0 \in \Gamma$, alors $I$ n'est pas bien définie
\end{itemize}
\textbf{Exemple :} (série 4 exercice 2) \\
Calculer
$$I = \int_{\Gamma} \frac{e^{z^2}}{(z-1)^2(z^2+4)}dz$$
lorsque :
\begin{itemize}
    \item[a)] $\Gamma$ est le cercle de rayon 1 centré en $z=1$ \\
    $I$ est de la forme :
    $$I = \int_{\Gamma} \frac{f(z)}{(z-z_0)^{n+1}}dz$$
    avec $f(z) = \frac{e^{z^2}}{z^2+4}$ qui est holomorphe sur $int(\Gamma) \cup \Gamma$, $z_0 = 1 \in int(\Gamma)$, $n=1$ \\
    On calcule la dérivée $f'(z)$ :
    $$f'(z) = \frac{2ze^{z^2}(z^2+3)}{(z^2+4)^2}$$
    Par la FIC (BIS) on obtient :
    \begin{align*}
        I &= 2\pi i f'(z_0) \\
        &= 2\pi i f'(1) \\
        &= 2\pi i \frac{2e(1+3)}{(1+4)^2} \\
        &= 2\pi i \frac{8e}{25} \\
        &= \frac{16e}{25}\pi i
    \end{align*}
    \item[b)] $\Gamma$ est le bord du rectangle $[-\frac{1}{2}, \frac{1}{2}] \times [0,4]$ \\
    On a $1 \notin int(\Gamma)$, on doit donc changer de forme \\
    $z^2+4 = (z+2i)(z-2i)$ \\
    $I$ est maintenant de la forme :
    $$I = \int_{\Gamma} \frac{f(z)}{z-z_0}dz$$
    avec $f(z) = \frac{e^{z^2}}{(z-1)^2(z+2i)}$ qui est holomorphe sur $int(\Gamma) \cup \Gamma$, $z_0 = 2i \in int(\Gamma)$ \\
    Par la FIC on obtient : 
    \begin{align*}
        I &= 2\pi i f(z_0) \\
        &= 2\pi i f(2i) \\
        &= 2\pi i \frac{e^{(2i)^2}}{(2i-1)^2(2i+2i)} \\
        &= \frac{2\pi i e^{-4}}{(-3-4i)4i} \\
        &= \frac{-\pi e^{-4}}{2(3+4i)}
    \end{align*}
    \item[c)] $\Gamma$ est le bord du rectangle $[-2,0] \times [-1,1]$
    $f(z) = \frac{e^{z^2}}{(z-1)^2(z^2+4)}$ est holomorphe sur $int(\Gamma) \cup \Gamma$ \\
    Par le théorème de Cauchy, on a $I = 0$
\end{itemize}
\textbf{Exemple :} (série 4 exercice 3b) \\
Soit $\Gamma \subset \mathbb{C}$ une courbe simple, fermée et régulière par morceaux. Discuter, en fonction de $\Gamma$, la valeur de l'intégrale :
$$I = \int_{\Gamma} \frac{5z^2-3z+2}{(z-1)^3}dz$$
$I$ est de la forme :
    $$I = \int_{\Gamma} \frac{f(z)}{(z-z_0)^{n+1}}dz$$
avec $f(z) = 5z^2-3z+2$ qui est holomorphe, $z_0 = 1 \in int(\Gamma)$, $n=2$ \\
On calcule la dérivée seconde : $f''(z) =10$
\begin{itemize}
    \item Si $z_0 \in int(\Gamma)$, alors par la FIC (BIS) on obtient :
    \begin{align*}
        I &= 2\pi i f''(z_0) \\
        &= 2\pi i f''(1) \\
        &= 20\pi i
    \end{align*}
    \item Si $z_0 \notin int(\Gamma) \cup \Gamma$, alors par le théorème de Cauchy on a $I = 0$
    \item Si $z_0 \in \Gamma$, alors $I$ n'est pas bien définie
\end{itemize}
\subsection{Nature des singularités et série de Laurent}
Trouver les singularités d’une fonction complexe, déterminer leur nature (donner l’ordre si il s’agit d’un pôle) et donner la série de Laurent/Taylor et son rayon de convergence. \\
(Série 5 exercice 2, série 6 exercice 2, série 7 exercice 1) \\
\\
\textbf{Méthodologie :}
\begin{itemize}
    \item On essaye de reconnaître une forme connue de série de Taylor d'une partie de $f(z)$
    \item On simplifie, change les indices
    \item Si la partie singulière de $L_{z_0}f(z)$ admet une infinité de termes, alors $z_0$ est une SEI
    \item Si la partie singulière de $L_{z_0}f(z)$ vaut $0$ (si la série de Laurent est égale à la série de Taylor), alors $z_0$ est un point régulier
    \item Si la partie singulière de $L_{z_0}f(z)$ commence à la puissance $-m$, alors $z_0$ est un pôle d'ordre $m$
    \item Le rayon de convergence $R > 0$ est le plus grand $\epsilon$ tel que
    $$f(z) = L_{z_0}f(z) \quad \text{, } \forall z \in \mathcal{B}_R(z_0) \setminus \{z_0\}$$
\end{itemize}
\textbf{Exemple :} (série 6 exercice 2a, 2d, 2g) \\
Pour chaque fonction $f(z)$, déterminer la série de Laurent au voisinage de $z_0$, préciser le rayon de convergence, déterminer la nature du point $z_0$ et trouver le résidu de la fonction en $z_0$
\begin{enumerate}
    \item[a)] $f(z) = z^2 e^{\frac{1}{z}}$, $z_0 = 0$ \\
    On connaît la série de Taylor de $e^z$ en $0$ :
    $$e^z = \sum_{n=0}^{\infty} = \frac{z^n}{n!}$$
    qui converge $\forall z \in \mathbb{C}$ \\
    La série de Laurent en $0$ est donc :
    \begin{align*}
        L_{z_0}f(z) &= z^2 \sum_{n=0}^{\infty} \frac{(\frac{1}{z})^{^n}}{n!} \\
        &= z^2 \sum_{n=0}^{\infty} \frac{z^{-n}}{n!} \\
        &= \sum_{n=0}^{\infty} \frac{z^{-n+2}}{n!} \\
        &= \sum_{m=-2}^{\infty} \frac{z^{-m}}{(m+2)!} \quad (m=n-2) \\
        &= z^2 + z + \frac{1}{2!} + \frac{z^{-1}}{3!} + \frac{z^{-2}}{4!} + ... \\
        &= z^2 + z + \frac{1}{2} + \sum_{n=1}^{\infty} \frac{z^{-n}}{(n+2)!}
    \end{align*}
    et elle converge $\forall z \neq 0$ \\
    la partie singulière de $L_{z_0}f(z)$ admet une infinité de termes donc $z_0$ est une SEI \\
    Le résidu de $f$ en $z_0$ est $Res_{z_0}(f) = \frac{1}{3!} = \frac{1}{6}$
    \item[d)] $f(z) = \frac{\sin(z)}{(z-\pi)^2}$ , $z_0 = \pi$ \\
    On a :
    $$f(z) = \frac{-\sin(z-\pi)}{(z-\pi)^2}$$
    On connaît la série de Taylor de $\sin(z)$ en $0$ :
    $$\sin(z) = \sum_{n=0}^{\infty} \frac{(-1)^n}{(2n+1)!}z^{(2n+1)}$$
    qui converge $\forall z \in \mathbb{C}$ \\
    La série de Laurent en $z_0$ est donc :
    \begin{align*}
        L_{z_0}f(z) &= \frac{-1}{(z-\pi)^2} \sum_{n=0}^{\infty} \frac{(-1)^n}{(2n+1)!}(z-\pi)^{(2n+1)} \\
        &= -\sum_{n=0}^{\infty} \frac{(-1)^n}{(2n+1)!}(z-\pi)^{(2n-1)} \\
        &= -(z-\pi)^{-1} - \sum_{n=1}^{\infty} \frac{(-1)^n}{(2n+1)!}(z-\pi)^{(2n-1)}
    \end{align*}
    et elle converge $\forall z \neq \pi$ \\
    la partie singulière de $L_{z_0}f(z)$ commence à la puissance $-1$ donc $z_0$ est un pôle d'ordre $1$ \\
    Le résidu de $f$ en $z_0$ est $Res_{z_0}(f) = -1$
    \item[g)] $f(z) = \frac{z^2+2z+1}{z+1}$, $z_0 = -1$ \\
    On a :
    $$f(z) = \frac{(z+1)^2}{z+1} = z+1$$
    La série de Laurent en $-1$ est donc :
    $$L_{z_0}(f) = 1 + z$$
    qui converge $\forall z \in \mathbb{C} : R = \infty$ \\
    la partie singulière de $L_{z_0}f(z)$ vaut $0$ donc $z_0$ est un point régulier \\
    Le résidu de $f$ en $z_0$ est $Res_{z_0}(f) = 0$
\end{enumerate}
\subsection{Résidu en un point}
Calculer le résidu d’une fonction complexe en un point. \\
(Série 7 exercices 2 et 4) \\
\\
\textbf{Méthodologie :}
\begin{itemize}
    \item Si $f(z)$ est une fonction rationnelle, c'est-à-dire si $f(z) = \frac{p(z)}{q(z)}$, alors on calcule le résidu de $f$ en $z_0$ avec la formule :
    $$Res_{z_0}(f) = Res_{z_0}\bigg{(}\frac{p}{q}\bigg{)} = \frac{p(z_0)}{q'(z_0)}$$
    \item Sinon on calcule le résidu de $f$ en $z_0$ avec la formule classique :
    $$\frac{1}{(m-1)!}\frac{d^{(m-1)}}{dz}(f(z)(z-z_0)^m)\bigg{|}_{z=z_0} = Res_{z_0}(f)$$
    avec $z_0$ un pôle d'ordre $m \geq 1$
\end{itemize}
\textbf{Exemple :} (série 7 exercice 2b) \\
Soit $\Gamma \subset \mathbb{C}$ une courbe simple, fermée et régulière par morceaux. Discuter, en fonction de $\Gamma$, la valeur de l'intégrale :
$$I = \int_{\Gamma} \frac{(z+1)^2}{(z-3)^3}dz$$
On a :
$$I = \int_{\Gamma} f(z)dz \quad \text{, avec } f(z) = \frac{z^2+2z+1}{(z-3)^3}$$
On remarque que $3$ est un pôle d'ordre $3$ \\
On calcule le résidu de $f$ en $z_0 = 3$ avec la formule à résidu :
\begin{align*}
    Res_3(f) &= \frac{1}{2!} \frac{d^2}{dz^2}(f(z)(z-3)^3) \bigg{|}_{z=3} \\
    &= \frac{1}{2} \frac{d^2}{dz^2}(z^2+2z+1) \bigg{|}_{z=3} \\
    &= 1
\end{align*}
\begin{itemize}
    \item Si $3 \in int(\Gamma)$, alors par le théorème des résidus on obtient :
    \begin{align*}
        I &= 2\pi i Res_3(f) \\
        &= 2\pi i
    \end{align*}
    \item Si $3 \notin int(\Gamma) \cup \Gamma$, alors par le théorème de Cauchy on a $I = 0$
    \item Si $3 \in \Gamma$, alors $I$ n'est pas bien définie
\end{itemize}
\textbf{Exemple :} (série 7 exercice 4) \\
Soit $\Gamma \subset \mathbb{C}$ une courbe simple, fermée et régulière par morceaux, contenue dans le disque de rayon $2$, centré en $z=0$. Discuter, en fonction de $\Gamma$, la valeur de l'intégrale :
$$I = \int_{\gamma} \tan(z)dz$$
On a :
$$\tan(z) = \frac{\sin(z)}{\cos(z)} = \frac{p(z)}{q(z)}$$
On remarque que $z_0 = \frac{\pi}{2}$ et $z_1 = -\frac{\pi}{2}$ sont des pôles d'ordre $1$ \\
On calcule les résidus de $f$ en $z_0$ et $z_1$ avec la formule de la fonction rationnelle  :
\begin{align*}
    Res_{z_0}(f) &= Res_{z_0}\bigg{(}\frac{p}{q}\bigg{)} = \frac{\sin(z_0)}{(\cos(z_0))'} = \frac{\sin(\frac{\pi}{2})}{-\sin(\frac{\pi}{2})} = -1 \\
    Res_{z_1}(f) &= Res_{z_1}\bigg{(}\frac{p}{q}\bigg{)} = \frac{\sin(z_1)}{(\cos(z_1))'} = \frac{\sin(-\frac{\pi}{2})}{-\sin(-\frac{\pi}{2})} = -1
\end{align*}
\begin{itemize}
    \item Si $z_0 \in int(\Gamma)$ et $z_1 \in int(\Gamma)$, alors par le théorème des résidus on obtient :
    \begin{align*}
        I &= 2\pi i (Res_{z_0}(f) + Res_{z_1}(f)) \\
        &= 2\pi i (-1-1) \\
        &= -4\pi i
    \end{align*}
    \item Si $z_0 \in int(\Gamma)$ et $z_1 \notin int(\Gamma) \cup \Gamma$, alors par le théorème des résidus on obtient :
    \begin{align*}
        I &= 2\pi i Res_{z_0}(f) \\
        &= -2\pi i
    \end{align*}
    \item Si $z_0 \notin int(\Gamma) \cup \Gamma$ et $z_1 \in int(\Gamma)$, alors par le théorème des résidus on obtient :
    \begin{align*}
        I &= 2\pi i Res_{z_0}(f) \\
        &= -2\pi i
    \end{align*}
    \item Si $z_0 \notin int(\Gamma) \cup \Gamma$ et $z_1 \notin int(\Gamma) \cup \Gamma$, alors par le théorème de Cauchy on a $I = 0$
    \item Si $z_0 \in \Gamma$ ou $z_1 \in \Gamma$, alors $I$ n'est pas bien définie
\end{itemize}
\subsection{Théorème des résidus et méthode du cercle}
Calculer des intégrales réelles à l’aide du théorème des résidus et de la méthode du cercle. \\
(Série 7 exercice 5, Série 8 exercice 1, 2 et 3) \\
\\
\textbf{Méthodologie :} On veut calculer $I = \int_0^{2\pi} f(\theta) d\theta$
\begin{itemize}
    \item Faire un changement de variable :
    \begin{align*}
        z &= e^{i\theta} \\
        dz &= ie^{i\theta}d\theta = izd\theta \Rightarrow d\theta = \frac{1}{iz}dz
    \end{align*}
    \item Exprimer les fonctions trigonométriques sous leur forme exponentielle, puis faire le changement de variable
    \item Calculer et simplifier $\widetilde{f}(z) = \frac{1}{iz}f(\theta)$, en fonction de $z$
    \item Déterminer les pôles de $\widetilde{f}(z)$
    \item Sélectionner seulement ceux se trouvant à l'intérieur du cercle unitaire $\Gamma$, c'est-à-dire les $z_i$ tels que $|z_i| < 1$, et déterminer leur ordre
    \item Calculer les résidus de chaque pôle sélectionné
    \item Calculer $I$ à l'aide du théorème des résidus
\end{itemize}
\textbf{Exemple :} (série 7 exercice 5) \\
Calculer
$$I = \int_0^{2\pi} \frac{\cos(\theta)\sin(2\theta)}{5 + 3\cos(\theta)}d\theta$$
On a :
$$f(\theta) = \frac{\cos(\theta)\sin(2\theta)}{5 + 3\cos(\theta)}$$
On procède au changement de variable :
\begin{align*}
    z &= e^{i\theta} \\
    dz &= ie^{i\theta}d\theta = izd\theta \Rightarrow d\theta
    \frac{1}{iz}dz
\end{align*}
On exprime les fonctions trigonométriques sous leur forme exponentielle :
\begin{align*}
    \cos(\theta) &= \frac{e^{i\theta} + e^{-i\theta}}{2} = \frac{1}{2}\bigg{(}z + \frac{1}{z}\bigg{)} \\
    \cos(2\theta) &= \frac{e^{i2\theta} + e^{-i2\theta}}{2} = \frac{1}{2}\bigg{(}z^2 + \frac{1}{z^2}\bigg{)} \\
    \sin(2\theta) &= \frac{e^{i2\theta} - e^{-i2\theta}}{2i} = \frac{1}{2i}\bigg{(}z^2 - \frac{1}{z^2}\bigg{)}
\end{align*}
Donc on a :
$$f(\theta) = \frac{\frac{1}{2}(z + \frac{1}{z}) \frac{1}{2i}(z^2 - \frac{1}{z^2})}{5 + \frac{3}{2} (z^2 + \frac{1}{z^2})}$$
On calcule $\widetilde{f}(z)$ :
\begin{align*}
    \widetilde{f}(z) &= \frac{1}{iz} f(\theta) \\
    &= \frac{1}{iz} \bigg{(} \frac{\frac{1}{2}(z + \frac{1}{z}) \frac{1}{2i}(z^2 - \frac{1}{z^2})}{5 + \frac{3}{2} (z^2 + \frac{1}{z^2})} \bigg{)} \\
    &= \frac{1}{iz} \bigg{(} \frac{\frac{1}{4i}(z + \frac{1}{z})(z^2 - \frac{1}{z^2})}{\frac{3}{2} (z^2 + \frac{10}{3} + \frac{1}{z^2})} \bigg{)} \\
    &= \frac{1}{iz} \frac{(z + \frac{1}{z})(z^2 - \frac{1}{z^2})}{4i} \frac{2}{3(z^2 + \frac{10}{3} + \frac{1}{z^2})} \\
    &= -\frac{(z + \frac{1}{z})(z^2 - \frac{1}{z^2})}{6z(z^2 + \frac{10}{3} + \frac{1}{z^2})} \\
    &= -\frac{z^3(z + \frac{1}{z})(z^2 - \frac{1}{z^2})}{z^36z(z^2 + \frac{10}{3} + \frac{1}{z^2})} \\
    &= -\frac{(z^2 + 1)(z^4 - 1)}{6z^2(z^4 + \frac{10}{3}z^2 + 1)} \\
    &= -\frac{(z^2 + 1)(z^4 - 1)}{6z^2(z^2+3)(z^2+\frac{1}{3})} \\
    &= \frac{-z^6-z^4+z^2+1}{6z^6+20z^4+6z^2}
\end{align*}
On détermine les pôles de $\widetilde{f}(z)$ :
\begin{align*}
    &6z^2(z^2+3)(z^2+\frac{1}{3}) = 0 \\
    \Rightarrow \quad &z_0 = 0 \text{, } z_1 = i\sqrt{\frac{1}{3}} \text{, } z_2 = -i\sqrt{\frac{1}{3}} \text{, } z_3 = i\sqrt{3} \text{, } z_4 = -i\sqrt{3}
\end{align*}
Les pôles se trouvant à l'intérieur du cercle unitaire sont $z_0 = 0$ qui est un pôle d'ordre $2$, $z_1 = i\sqrt{\frac{1}{3}}$ qui est un pôle d'ordre $1$ et $z_2 = -i\sqrt{\frac{1}{3}}$ qui est un pôle d'ordre $1$ \\
On calcule les résidus de $f$ en $z_0$, $z_1$ et $z_2$ :
\begin{align*}
    Res_0(\widetilde{f}) &= \frac{d}{dz} \bigg{(} \widetilde{f}(z)(z-0)^2 \bigg{)} \bigg{|}_{z=0} = \frac{d}{dz} \bigg{(} \frac{-z^6-z^4+z^2+1}{6z^6+20z^4+6z^2} \bigg{)} \bigg{|}_{z=0} = 0 \\
    Res_{i\sqrt{\frac{1}{3}}}(\widetilde{f}) &= \widetilde{f}(z)\bigg{(}z-i\sqrt{\frac{1}{3}}\bigg{)} \bigg{|}_{z=i\frac{1}{\sqrt{3}}} = -\frac{(z^2 + 1)(z^4 - 1)}{6z^2(z^2+3)(z+i\frac{1}{\sqrt{3}})} \bigg{|}_{z=i\frac{1}{\sqrt{3}}} = i\frac{1}{6\sqrt{3}} \\
    Res_{-i\sqrt{\frac{1}{3}}}(\widetilde{f}) &= \widetilde{f}(z)\bigg{(}z+i\sqrt{\frac{1}{3}}\bigg{)} \bigg{|}_{z=-i\frac{1}{\sqrt{3}}} = -\frac{(z^2 + 1)(z^4 - 1)}{6z^2(z^2+3)(z-i\frac{1}{\sqrt{3}})} \bigg{|}_{z=-i\frac{1}{\sqrt{3}}} = -i\frac{1}{6\sqrt{3}}
\end{align*}
Par le théorème des résidus, on obtient :
\begin{align*}
    I &= 2\pi i (Res_0 + Res_{i\sqrt{\frac{1}{3}}} + Res_{-i\sqrt{\frac{1}{3}}}) \\
    &= 2 \pi i (0 + i\frac{1}{6\sqrt{3}} -i\frac{1}{6\sqrt{3}}) \\
    &= 0
\end{align*}

\subsection{Théorème des résidus et méthode du demi-cercle}
Calculer des intégrales réelles à l’aide du théorème des résidus et de la méthode du demi-cercle. \\
(Série 8 exercices 4 et 5, série 9 exercice 1) \\
\\ 
\textbf{Méthodologie :} On veut calculer $I = \int_{-\infty}^{\infty} f(x) dx$
\begin{itemize}
    \item Si $f(x)$ est de la forme 
    $$\int_{-\infty}^{\infty}\frac{N(x)}{D(x)}e^{i\alpha x} dx$$
    avec $N(x), D(x)$ des polynômes tels que $deg(D)-deg(N) \geq 2$ et $D(x) = \beta(x) + \delta$, $\delta \in \mathbb{R}$\\
    Alors on choisit notre demi-cercle $\Gamma_R$ de rayon $R$ en fonction du signe de $\alpha$ :
    \begin{itemize}
        \item Si $\alpha > 0 : \Gamma_R$ le demi-cercle supérieur
        \item Si $\alpha < 0 : \Gamma_R$ le demi-cercle inférieur (\faExclamationTriangle\hspace{1pt} orientation)
        \item Si $\alpha = 0 :$ peu importe
    \end{itemize}
    \item On pose $z = re^{i\theta}$ et on montre avec l'inégalité triangulaire inverse ($^{\Delta \text{inv}})$ que :
    $$|f(z)| \leq \frac{N(z)}{|\beta(z)| - \delta}$$
    \item On note $\Gamma_R = C_R \cup L_R$, avec $C_R$ le demi-cercle ouvert supérieur ou inférieur, et $L_R$ le segment réel $[-R;R]$
    \item On choisit la paramétrisation $\gamma(t)$ de $C_R$ en fonction de notre demi-cercle :
    \begin{align*}
        C_R = \{\gamma(t) = re^{it} \mid t : 0 \to \pi\} \quad &\text{si } \Gamma_R \text{ est le demi-cercle supérieur} \\
        C_R = \{\gamma(t) = re^{it} \mid t : \pi \to 2\pi\} \quad &\text{si } \Gamma_R \text{ est le demi-cercle inférieur}
    \end{align*}
    \item On montre à l'aide de la proposition 5.3 et de l'inégalité prouvée au 2ème point que :
    $$\lim_{R \to \infty}\bigg{|}\int_{C_R} f(z)dz\bigg{|} = 0$$
    \item On détermine les pôles de $f(z)$, on conserve seulement ceux à l'intérieur de $\Gamma_R$ et on détermine leur ordre
    \item On calcule les résidus des pôles conservés
    \item On calcule $\int_{\Gamma_R} f(z) dz$ avec le théorème des résidus
    \item On connaît maintenant la valeur de
    $$\int_{\Gamma_R} f(z) dz = \underbrace{\int_{C_R} f(z) dz}_{\overset{R \to \infty}{\longrightarrow} 0} \pm \int_{L_R} f(z) dz$$
    avec le signe du haut si $\Gamma_R$ est le demi-cercle supérieur et le signe du bas si $\Gamma_R$ est le demi-cercle inférieur \\
    De plus, on a
    $$\int_{L_R} f(z) dz = \int_{\mp R}^{\pm R} f(z) dz \overset{R \to \infty}{\longrightarrow} \pm \int_{-\infty}^{\infty} f(z) dz = \pm \int_{-\infty}^{\infty} f(x) dx = \pm I$$
    Donc :
    $$I = \pm \int_{\Gamma_R} f(z) dz$$
\end{itemize}
\textbf{Exemple :} (série 9 exercice 1) \\
Calculer
$$\int_{-\infty}^{\infty} \frac{e^{-2ix}}{x^2+2}dx, \quad \int_{-\infty}^{\infty} \frac{\cos(2x)}{x^2+2}dx, \quad \int_{-\infty}^{\infty} \frac{\sin(2x)}{x^2+2}dx$$
On a $f(x)$ de la forme :
$$\int_{-\infty}^{\infty}\frac{N(x)}{D(x)}e^{i\alpha x} dx$$
avec $\alpha = -2$, $N(x) = 1$, $D(x) = \beta(x) + \delta$, $\beta(x) = x^2$, $\delta = 2$ et on a bien $deg(D)-deg(N) \geq 2$ \\
On choisit $\Gamma_R$ le demi-cercle inférieur car $\alpha < 0$ \\
On pose $z = re^{i\theta}$ \\
On montre $|f(z)| \leq \frac{N(z)}{|\beta(z)| - \delta}$ :
\begin{align*}
    |z^2+2| &\overset{\Delta\text{inv}}{\geq} \big{|} |z^2|-|2| \big{|} \\
    &= r^2 - 2 \\
    \Rightarrow \quad \bigg{|} \frac{e^{-2iz}}{z^2 + 2} \bigg{|} &\leq \bigg{|} \frac{e^{-2iz}}{r^2 - 2} \bigg{|} \\
    &= \frac{1}{r^2 - 2}
\end{align*}
On note $\Gamma_R = C_R \cup L_R$, avec $C_R$ le demi-cercle ouvert inférieur, et $L_R$ le segment réel $[-R;R]$ \\
On choisit la paramétrisation $\gamma(t)$ de $C_R$ :
$$C_R = \{\gamma(t) = re^{it} \mid t : \pi \to 2\pi\}$$
On montre $\lim_{R \to \infty} |\int_{C_R} f(z) dz| = 0$ :
\begin{align*}
    \bigg{|}\int_{C_R} f(z)dz\bigg{|} &\leq \int_\pi^{2\pi} |f(\gamma(t))||\gamma'(t)|dt \\
    &= \int_\pi^{2\pi} \bigg{|} \frac{e^{-2ire^{it}}}{r^2e^{2it}+2} \bigg{|} |ire^{it}| dt \\
    &\leq \int_\pi^{2\pi} \frac{r}{r^2 - 2} dt \\
    &= \frac{\pi r}{r^2 - 2} \\
    &\overset{R \to \infty}{\longrightarrow} 0
\end{align*}
On a :
$$f(z) = \frac{e^{-2iz}}{z^2+2}$$
Les pôles de $f(z)$ sont $z_0 = i\sqrt{2}$ et $z_1 = -i\sqrt{2}$ \\
Seul $z_1$ est à l'intérieur de $\Gamma_R$ et c'est un pôle d'ordre $1$ \\
On calcule le résidu de $f$ en $z_1$ :
$$Res_{-i\sqrt{2}}(f) = Res_{-i\sqrt{2}}\bigg{(}\frac{e^{-2iz}}{z^2+2}\bigg{)}= \frac{e^{-2iz}}{(z^2+2)'} \bigg{|}_{z=-i\sqrt{2}} = \frac{e^{-2i(-i\sqrt{2})}}{2(-i\sqrt{2})} = -\frac{e^{-2\sqrt{2}}}{2i\sqrt{2}}$$
Par le théorème des résidus on obtient :
$$\int_{\Gamma_R} f(z) dz = 2\pi i Res_{z_1} = -\frac{\pi e^{-2\sqrt{2}}}{\sqrt{2}}$$
Vu que :
$$
\begin{cases}
    \int_{\Gamma_R} f(z) dz &= \int_{C_R} f(z) dz - \int_{L_R} f(z) dz \\
    \int_{L_R} f(z) dz &= \int_{R}^{-R} f(z) dz \overset{R \to \infty}{\longrightarrow} -\int_{-\infty}^{\infty} f(z) dz \\
    \int_{C_R} f(z) dz &\overset{R \to \infty}{\longrightarrow} 0
\end{cases}
$$
avec le signe $-$ car $\Gamma_R$ est le demi-cercle inférieur \\
Alors :
$$I = -\int_{\Gamma_R} f(z) dz = \frac{\pi e^{-2\sqrt{2}}}{\sqrt{2}}$$
On a aussi :
$$I = \int_{-\infty}^{\infty} \frac{e^{-2ix}}{x^2+2}dx = \int_{-\infty}^{\infty} \frac{\cos(2x)}{x^2+2}dx - i \int_{-\infty}^{\infty} \frac{\sin(2x)}{x^2+2}dx$$
Donc :
\begin{align*}
    \int_{-\infty}^{\infty} \frac{\cos(2x)}{x^2+2}dx &= \frac{\pi e^{-2\sqrt{2}}}{\sqrt{2}}\\
    \int_{-\infty}^{\infty} \frac{\sin(2x)}{x^2+2}dx &= 0
\end{align*}


\subsection{Transformée de Laplace}
Calculer la transformée de Laplace d’une fonction à l’aide d’un calcul direct ou des tables de transformées de Laplace et des propriétés de la transformée de Laplace. \\
(Série 9 exercices 2, 3 et 4, série 10 exercice 1)

\subsection{Transformée inverse de Laplace}
Calculer la transformée inverse de Laplace d’une fonction à l'aide des résidus ou des tables de la transformée de Laplace. \\
(Série 10 exercices 2, 3 et 4) \\
\\
\textbf{Méthodologie :} (pour la méthode avec les résidus)
\begin{itemize}
    \item On pose $\widetilde{F}(z) = e^{tz}F(z)$
    \item On détermine les pôles de $\widetilde{F}(z)$
    \item On calcule les résidus de $\widetilde{F}(z)$ en chaque pôle
    \item On utilise le corrolaire 6.5 pour calculer la transformée inverse de Laplace de $F$
\end{itemize}
\textbf{Exemple :} (série 10 exercice 4)
Déterminer la transformée inverse de Laplace de
$$F(z) = \frac{1}{z^3-9z}$$
de 2 manières différentes :
\begin{itemize}
    \item[a)] En utilisant la table des transformées :
    $$F(z) = \frac{1}{z(z^2-9)} = \frac{\alpha}{z} + \frac{\beta z + \gamma}{z^2+9} = \frac{\alpha z^2 - 9 \alpha + \beta z^2 + \gamma z}{z(z^2-9)}$$
    $$
    \begin{cases}
        z^2 &: \alpha + \beta = 0 \Rightarrow \beta = \frac{1}{9} \\
        z &: \alpha = 0 \\
        1 &: -9 \alpha = 1 \text{ } \Rightarrow \alpha = -\frac{1}{9}
    \end{cases}
    $$
    $$F(z) = -\frac{1}{9} \frac{1}{z} + \frac{1}{9} \frac{z}{z^2-3^2}$$
    $$\Rightarrow \quad \mathcal{L}^{-1}(F)(t) = -\frac{1}{9} + \frac{1}{9} \cosh(3t)$$
    \item[b)] En utilisant les résidus : \\
    On pose
    $$\widetilde{F}(z) = e^{tz}F(z) = \frac{e^{tz}}{z(z^2-9)} = \frac{e^{tz}}{z(z+3)(z-3)}$$
    On détermine les pôles de $\widetilde{F}(z)$ :
    \begin{align*}
        z_0 = 0 \quad &\text{pôle d'ordre 1} \\
        z_1 = 3 \quad &\text{pôle d'ordre 1} \\
        z_2 = -3 \quad &\text{pôle d'ordre 1}
    \end{align*}
    On calcule les résidus de $\widetilde{F}(z)$ en $z_0$, $z_1$ et $z_2$ :
    \begin{align*}
        Res_0(\widetilde{F}) &= \widetilde{F}(z)z \big{|}_{z=0} = \frac{e^{tz}}{z^2-9} \bigg{|}_{z=0} = -\frac{1}{9} \\
        Res_3(\widetilde{F}) &= \widetilde{F}(z)(z-3) \big{|}_{z=3} = \frac{e^{tz}}{z(z+3)} \bigg{|}_{z=3} = \frac{e^{3t}}{18} \\
        Res_{-3}(\widetilde{F}) &= \widetilde{F}(z)(z+3) \big{|}_{z=-3} = \frac{e^{tz}}{z(z-3)} \bigg{|}_{z=-3} = \frac{e^{-3t}}{18}
    \end{align*}
    Par le corrolaire 6.5 on obtient :
    \begin{align*}
        \mathcal{L}^{-1}(F)(t) &= \sum_{i} Res_{z_i}(\widetilde{F}(z)) \\
        &= -\frac{1}{9} + \frac{e^{3t}}{18} + \frac{e^{-3t}}{18} \\
        &= -\frac{1}{9} + \frac{1}{9} \cosh(3t)
    \end{align*}
\end{itemize}
\subsection{EDO}
Résoudre des EDOs (problèmes de Cauchy ou Sturm-Liouville) à l’aide de la transformée de Laplace. \\
(Série 10 exercices 5 et 6, série 11 exercices 1 et 5) \\
\\
\textbf{Méthodologie :}
\begin{itemize}
    \item Appliquer la transformée de Laplace à chaque fonctions, dérivées, convolutions de l'équation
    \item Utiliser les propriétés de la transformée de Laplace pour faire apparaître des formes connues de la table des transformées
    \item Utiliser la table des transformées de Laplace pour résoudre l'équation
\end{itemize}
\textbf{Exemple :} (série 11 exercice 1) \\
Utiliser les méthodes de la transformée de Laplace et la table  des transformées pour trouver une solution $y(t)$ de l'équation intégrale :
$$
\begin{cases}
    14y'(t) + 13y(t) + \int_{0}^{t} y(s) e^{\frac{13}{14}(t-s)} ds = e^{\frac{13}{14}t} \quad \text{, } t \geq 0 \\
    y(0) = 0
\end{cases}
$$
On pose $f(t) = e^\frac{13}{14}t$ \\
On a donc :
\begin{align*}
    &\Rightarrow 14 \mathcal{L}(y') + \mathcal{L}(y) + \mathcal{L}(y \ast f) = \mathcal{L}(f) \\
    &\Rightarrow 14(z \mathcal{L}(y) - y(0)) + 13 \mathcal{L}(y) + \mathcal{L}(y) \mathcal{L}(f) = \mathcal{L}(f) \\
    &\Rightarrow \mathcal{L}(y)\bigg{(}14z + 13 + \frac{1}{z - \frac{13}{14}}\bigg{)} = \frac{1}{z - \frac{13}{14}} \\
    &\Rightarrow \mathcal{L}(y) = \frac{\frac{1}{z-\frac{13}{14}}}{14z + 13 + \frac{1}{z - \frac{13}{14}}} \\
    &\Rightarrow \mathcal{L}(y) = \frac{1}{z-\frac{13}{14}} \frac{1}{\frac{14z^2 -13z +13z -\frac{13^2}{14} +1}{z-\frac{13}{14}}} \\
    &\Rightarrow \mathcal{L}(y) = \frac{1}{z-\frac{13}{14}} \frac{z-\frac{13}{14}}{14z^2 - \frac{155}{14}} \\
    &\Rightarrow \mathcal{L}(y) = \frac{\frac{1}{14}}{z^2 - \frac{155}{196}} \\
    &\Rightarrow \mathcal{L}(y) = \frac{1}{14} \sqrt{\frac{196}{155}} \frac{\sqrt{\frac{155}{196}}}{z^2 - \frac{155}{196}} \\
    &\Rightarrow \mathcal{L}(y) = \frac{1}{\sqrt{155}} \frac{\sqrt{\frac{155}{196}}}{z^2 - \frac{155}{196}}
\end{align*}
Et donc on a :
$$y(t) = \frac{1}{\sqrt{155}} \sinh \bigg{(} \sqrt{\frac{155}{196}t} \bigg{)}$$
\subsection{EDP et la méthode de séparation des variables}
Résoudre des EDPs à l’aide de la méthode de séparation des variables. \\
(Série 12 exercices 1, 2, 3, 4 et 5) \\
\\
\textbf{Méthodologie :}
\begin{itemize}
    \item On pose $u(x,t) = \phi(x)\psi(t)$
    \item On reconnaît un problème de Sturm-Liouville avec la variable $\phi(x)$ et une autre équation différentielle connue avec la variable $\psi(t)$
    \item On trouve les solutions non-triviales $\phi_n(x)$ du problème de Sturm-Liouville, qui dépendent de certains coefficients
    \item On trouve les solutions $\psi_n(t)$ de cette deuxième équation différentielle
    \item On a les $u_n(x,t) = \phi_n(x)\psi_n(t)$, qui dépendent de certains coefficients, qui sont solutions du problème général (sans la contrainte des conditions initiales)
    \item On utilise le fait que la somme des $u_n(x,t)$ est aussi solution
    \item On détermine les coefficients précédents avec la ou les conditions initiales
    \item On obtient la solution $u(x,t)$
\end{itemize}
\textbf{Exemple :} (série 12 exercice 1) \\
Trouver la solution $u(x,t)$ du problème suivant :
$$
\begin{cases}
    u_t(x,t) = u_{xx}(x,t) \quad &\text{, } x \in ]0;\pi[ \text{ et } t > 0 \\
    u_x(0,t) = u_x(\pi,t) = 0 \quad &\text{, } t > 0 \\
    u(x,0) = \cos(2x) \quad &\text{, } x \in ]0;\pi[
\end{cases}
$$
On pose $u(x,t) = \phi(x)\psi(t)$ et on obtient :
\begin{align*}
    \begin{cases}
        \phi(x)\psi'(t) = \phi''(x)\psi(t) \\
        \phi'(0) = \psi(t) = \phi'(\pi)\psi(t) = 0 \\
        \phi(x)\psi(0) = \cos(2x)
    \end{cases}
    \Rightarrow \quad
    \begin{cases}
        \frac{\psi'(t)}{\psi(t)} \frac{\phi''(x)}{\phi(x)} = -\lambda \text{ (cst)} \\
        \phi'(0) = \phi'(\pi) = 0
    \end{cases}
    \Rightarrow \quad
    &\begin{cases}
            \phi''(x) + \lambda \phi(x) = 0 \tag{1} \\
            \phi'(0) = \phi'(\pi) = 0
    \end{cases} \\
    &\begin{aligned}
        \quad \psi'(t) + \lambda \psi(t) = 0 \qquad \qquad \qquad \text{ }(2)
    \end{aligned}
\end{align*}
L'équation $(1)$ se résout avec Sturm-Liouville 2 \\
Le problème de Sturm-Liouville 2 admet des solutions $\phi_n(x)$ non-triviales : \\
$$\text{si } \lambda = \bigg{(} \frac{n\pi}{\pi} \bigg{)}^2 = n^2 \quad \text{alors} \quad \phi_n(t) = \alpha_n \cos \bigg{(} \frac{n\pi}{\pi}x \bigg{)} = \alpha_n \cos(nx) \text{, } \alpha_n \in \mathbb{R}$$
On pose donc $\lambda = n^2$ \\
L'équation $(2)$ admet les solutions :
$$\psi_n(t) = e^{-\lambda t} = e^{-n^2t}$$
On a donc :
$$u_n(x,t) = \phi_n(x)\psi_n(t) = \alpha_n\cos(nx)e^{-n^2t}$$
qui est une solution pour tout $\mathbb{N}^*$ des 2 premières équations du problème \\
La somme des $u_n(x,t)$ est aussi solution :
$$u(x,t) = \sum_{n=0}^{\infty} u_n(x,t) = \frac{\alpha_0}{2} + \sum_{n=1}^{\infty} \alpha_n \cos(nx)e^{-n^2t}$$
On trouve les coefficients $\alpha_n \in \mathbb{R}$ avec la condition initiale $u(x,0) = \cos(2x)$ :
$$u(x,0) = \cos(2x) = \frac{\alpha_0}{2} + \sum_{n=1}^{\infty} \alpha_n \cos(nx) = \mathcal{F}_c(f)(x) \quad \text{la transformée de Fourier en cosinus}$$
On a donc nécessairement :
$$\alpha_2 = 1 \text{ et } \alpha_n = 0 \text{ } \forall n \neq 2$$
On obtient donc la solution :
$$u(x,t) = \cos(2x) e^{-4t}$$

\subsection{EDP et la méthode par transformée de Fourier}
Résoudre des EDPs à l’aide de la méthode par transformée de Fourier. \\
(Série 13 exercices 1, 2, 3, 4) \\
\\
\textbf{Méthodologie :}
\begin{itemize}
    \item Poser $$v(\alpha,y) = \frac{1}{2\pi} \int_{-\infty}^{\infty} u(x,y) e^{-i\alpha x}dx = \mathcal{F}_x(u)(\alpha)$$
    \item Appliquer la transformée de Fourier aux équations du problème
    \item Reconnaître une forme d'équation différentielle connue pour $v(\alpha,y)$ et trouver la solutions
    \item Calculer la transformée inverse $u(x,y) = \mathcal{F}_x^{-1}(v(\alpha,y))(x)$ en réduisant à des formes connues de la table
\end{itemize}
\textbf{Exemple :} (série 13 exercice 1) \\
Soit $\Omega = \{(x,y) \in \mathbb{R}^2 \mid y > 0\}$. Résoudre à l'aide de la transformée de Fourier le problème suivant :
$$
\begin{cases}
    \Delta u(x,y) = u_{xx}(x,y) + u_{yy}(x,y) = 0 \\
    u(x,0) = f(x) = \frac{8x^2}{(1+x^2)^2} \\
    \lim_{y \to \infty} u(x,y) = 0
\end{cases}
$$
On pose :
\begin{align*}
    v(\alpha,y) &= \frac{1}{2\pi} \int_{-\infty}^{\infty} u(x,y) e^{-i\alpha x}dx = \mathcal{F}_x(u)(\alpha) \\
    \widehat{f}(\alpha) &= 2\sqrt{2\pi} (1-|\alpha|) e^{-|\alpha|}
\end{align*}
On applique la tranformée de Fourier aux équations du problème :
\begin{align*}
    \Rightarrow
    \begin{cases}
        \mathcal{F}_x(u_{xx})(\alpha) = -\mathcal{F}_x(u_{yy})(\alpha) \\
        \mathcal{F}_x(u(x,0))(\alpha) = \mathcal{F}_x(f)(\alpha) \\
        \lim_{y \to \infty} \mathcal{F}_x(u)(\alpha) = 0
    \end{cases}
    \Rightarrow
    \begin{cases}
        (i\alpha)^2 v(\alpha,y) = -v_{yy}(\alpha,y) \\
        v(\alpha,0) = \widehat{f}(\alpha) \\
        \lim_{y \to \infty} v(\alpha,y) = 0
    \end{cases}
    \Rightarrow
    \begin{cases}
        v_{yy}(\alpha,y) - \alpha^2 v(\alpha,y) = 0 \tag{3} \\
        v(\alpha,0) = \widehat{f}(\alpha) \\
        \lim_{y \to \infty} v(\alpha,y) = 0
    \end{cases}
\end{align*}
La solution de $(3)$ est donnée par :
$$v(\alpha,y) = \widehat{f}(\alpha)e^{-|\alpha|y} = 2\sqrt{2\pi}(1-|\alpha|) e^{-|\alpha|(1+y)}$$
On calcule la transformée inverse en réduisant à des formes connues de la table :
\begin{align*}
    u(x,y) &= \mathcal{F}_x^{-1}(v(\alpha,y))(x) \\
    &= \mathcal{F}_x^{-1}(2\sqrt{2\pi}(1-|\alpha|) e^{-|\alpha|(1+y)})(x) \\
    &= \mathcal{F}_x^{-1} \bigg{(} 2\sqrt{2\pi} \bigg{(} \frac{1+y}{1+y} - |\alpha| \bigg{)} e^{-|\alpha|(1+y)} \bigg{)}(x) \\
    &= \mathcal{F}_x^{-1} \bigg{(} 2\sqrt{2\pi} \bigg{(} \frac{1}{1+y} + \frac{y}{1+y} - |\alpha| \bigg{)} e^{-|\alpha|(1+y)} \bigg{)}(x) \\
    &= \mathcal{F}_x^{-1} \bigg{(} 2\sqrt{2\pi} \bigg{(} \frac{1}{1+y} - |\alpha| \bigg{)} e^{-|\alpha|(1+y)} \bigg{)}(x) + \mathcal{F}_x^{-1} \bigg{(} 2\sqrt{2\pi} \bigg{(} \frac{y}{1+y}\bigg{)} e^{-|\alpha|(1+y)} \bigg{)}(x) \\
    &= 2\mathcal{F}_x^{-1} \bigg{(} \sqrt{2\pi} \bigg{(} \frac{1}{1+y} - |\alpha| \bigg{)} e^{-|\alpha|(1+y)} \bigg{)}(x) + 4y \mathcal{F}_x^{-1} \bigg{(} \sqrt{\frac{\pi}{2}} \frac{e^{-|\alpha|(1+y)}}{1+y} \bigg{)}(x) \\
    &= 2 \frac{4x^2}{((1+y)^2+x^2)} + 4y \frac{1}{x^2+(1+y)^2} \\
    &= 4 \frac{y(1+y)^2+x^2(2+y)}{((1+y)^2+x^2)^2}
\end{align*}


