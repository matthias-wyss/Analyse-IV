\section{Fonctions holomorphes}
\subsection{Fonctions complexes}
$f$ : $\mathbb{C}\to\mathbb{C}$\\
$z\mapsto f(z) = f(x+iy)$\\
On note :
$$f(z)=f(x+iy)=u(x,y)+iv(x,y)=\Re(f(x+iy))+i\Im(f(x+iy))$$
avec $u,v\in \mathbb{R}^2\to\mathbb{R}$

\subsection{Limite}
\textbf{Définition :} Soit $f$ : $\mathbb{C}\to\mathbb{C}$, $z_0\in\mathbb{C}$
\begin{enumerate}
    \item La limite de $f(z)$ quand $z$ tend vers $z_0$ est l'unique nombre $L\in\mathbb{C}$ tel que\\
    $\forall\epsilon>0$, $\exists\delta>0$ tel que si $|z-z_0|<\delta$, alors $|f(z)-L|<\epsilon$\\
    On note $L = \lim_{z \to z_0} f(z)$
    \item $f$ est continue en $z_0$ si $\lim_{z \to z_0}f(z) = f(z_0)$
\end{enumerate}

\subsection{Fonctions holomorphes}
\textbf{Définition :} Soit $\Omega\subset\mathbb{C}$ ouvert, $f$ : $\Omega\to\mathbb{C}$, $z_0\in\Omega$
\begin{enumerate}
    \item $f$ est holomorphe en $z_0$ si la limite suivante existe :
    $$\lim_{h\to0}\frac{f(z_0+h)-f(z_0)}{h}=f'(z_0)$$
    \item $f$ est holomorphe sur $\Omega$ si $f$ est holomorphe sur tout $z_0\in\Omega$
\end{enumerate}

\subsection{Équations de Cauchy-Riemann}
\textbf{Théorème :} Soit $\Omega\subset\mathbb{C}$ ouvert, $f=u+iv$ : $\Omega\to\mathbb{C}$\\
$f$ est holomorphe sur $\Omega$ si et seulement si les équations de Cauchy-Riemann sont satisfaites :
$$
\begin{cases}
    \frac{\partial u}{\partial x} = \frac{\partial v}{\partial y} \quad &(u_x=v_y) \\
    \frac{\partial v}{\partial x} = -\frac{\partial u}{\partial y} \quad &(v_x=-u_y) \quad \text{, }u,v\in\mathcal{C}^1(\Omega) 
\end{cases}
$$
De plus, on a :
\begin{align*}
    f'(z)&=u_x+iv_x\\
    &=v_y-iu_y
\end{align*}

\subsection{Remarques}
\begin{enumerate}
    \item Les règles de dérivations dans $\mathbb{R}$ sont aussi vraies dans $\mathbb{C}$ $(+,-,\cdot,\div,\circ)$
    \item La limite de $f'(z_0)$ s'écrit aussi
    $$\lim_{z\to z_0}\frac{f(z)-f(z_0)}{z-z_0}=f'(z_0)$$
    \item Cette limite est beaucoup plus "forte" que la limite dans $\mathbb{R}$\\
    Dans $\mathbb{R}$, on a 2 chemins (limites) pour avoir $f'(x_0)$ (limites à gauche et à droite).\\
    Dans $\mathbb{C}$, on a une infinité de chemins, c'est une notion plus puissante.
    \item On verra que $f$ holomorphe $\Rightarrow u,v\in\mathcal{C}^{\infty}$
    \item Les équations de Cauchy-Riemann disent que :
    \begin{itemize}
        \item $u(x,y)$ est un potentiel de $(v_y;-v_x)$
        \item $v(x,y)$ est un potentiel de $(-u_y;u_x)$
    \end{itemize}
\end{enumerate}

\subsection{Exponentielle complexe}
\textbf{Définition (Identité d'Euler) :}
\begin{align*}
    e^z &= \sum^\infty_{n=0}\frac{z^n}{n!}\\
    &= e^{x+iy}\\
    &= e^x e^{iy}\\
    &= e^x(\cos y + i\sin y)
\end{align*}
\textbf{Théorème :}
\begin{itemize}
    \item $e^z$ holomorphe dans $\mathbb{C}$ et $(e^z)' = e^z$
    \item $|e^z| = e^x$, $|e^{iy}|=1$
    \item $e^{z+2\pi ni} = e^z$, $\forall n \in \mathbb{Z}$
\end{itemize}

\subsection{Logarithme complexe}
\textbf{Définition :}
La fonction $\log$ complexe est définie par
$$\log : \mathbb{C}\setminus\{0\}\to\mathbb{C}$$
$$\log(z) := \underbrace{log(|z|)}_{\in\mathbb{R}} + i\underbrace{\arg(z)}_{\in[-\pi;\pi]}$$
avec le log réel et la détermination principal de l'argument. \\
\textbf{Théorème :}
\begin{itemize}
    \item $\log(e^z) = z$, si $\Im(z) \in ]-\pi;\pi[$ 
    \item $e^{\log(z)} = z$, si $z\neq 0$
    \item $\log(z)$ est holomorphe dans $\mathbb{C}\setminus\{z=x+iy\mid x \leq 0, y=0\}$ \\
    ($\mathbb{C}$ sans le demi-axe des réels négatifs)
\end{itemize}
\textbf{Remarque :} $\log(z)$ est en fait discontinue sur l'axe des réels négatifs :
\begin{align*}
    \lim_{t \to 0^+}\log(-1+it) = \lim_{t \to 0^+}(\underbrace{\log(|-1+it|)}_{\to \log(1)=0} + i\underbrace{\arg(-1+it)}_{\to \pi}) &= \underbrace{i\pi}_{_{_{\neq}}} \\
    \lim_{t\to 0^+}\log(-1-it) = \lim_{t \to 0^+}(\underbrace{\log(|-1-it|)}_{\to 0}) + i\underbrace{\arg(-1-it)}_{\to -\pi}) &= \overbrace{-i\pi}
\end{align*}
$$\Rightarrow \log(z)\text{ n'est pas continu !}$$