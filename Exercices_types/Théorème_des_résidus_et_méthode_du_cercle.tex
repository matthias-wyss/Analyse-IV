\subsection{Théorème des résidus et méthode du cercle}
Calculer des intégrales réelles à l’aide du théorème des résidus et de la méthode du cercle. \\
(Série 7 exercice 5, Série 8 exercice 1, 2 et 3) \\
\\
\textbf{Méthodologie :} On veut calculer $I = \int_0^{2\pi} f(\theta) d\theta$
\begin{itemize}
    \item Faire un changement de variable :
    \begin{align*}
        z &= e^{i\theta} \\
        dz &= ie^{i\theta}d\theta = izd\theta \Rightarrow d\theta = \frac{1}{iz}dz
    \end{align*}
    \item Exprimer les fonctions trigonométriques sous leur forme exponentielle, puis faire le changement de variable
    \item Calculer et simplifier $\widetilde{f}(z) = \frac{1}{iz}f(\theta)$, en fonction de $z$
    \item Déterminer les pôles de $\widetilde{f}(z)$
    \item Sélectionner seulement ceux se trouvant à l'intérieur du cercle unitaire $\Gamma$, c'est-à-dire les $z_i$ tels que $|z_i| < 1$, et déterminer leur ordre
    \item Calculer les résidus de chaque pôle sélectionné
    \item Calculer $I$ à l'aide du théorème des résidus
\end{itemize}
\textbf{Exemple :} (série 7 exercice 5) \\
Calculer
$$I = \int_0^{2\pi} \frac{\cos(\theta)\sin(2\theta)}{5 + 3\cos(\theta)}d\theta$$
On a :
$$f(\theta) = \frac{\cos(\theta)\sin(2\theta)}{5 + 3\cos(\theta)}$$
On procède au changement de variable :
\begin{align*}
    z &= e^{i\theta} \\
    dz &= ie^{i\theta}d\theta = izd\theta \Rightarrow d\theta
    \frac{1}{iz}dz
\end{align*}
On exprime les fonctions trigonométriques sous leur forme exponentielle :
\begin{align*}
    \cos(\theta) &= \frac{e^{i\theta} + e^{-i\theta}}{2} = \frac{1}{2}\bigg{(}z + \frac{1}{z}\bigg{)} \\
    \cos(2\theta) &= \frac{e^{i2\theta} + e^{-i2\theta}}{2} = \frac{1}{2}\bigg{(}z^2 + \frac{1}{z^2}\bigg{)} \\
    \sin(2\theta) &= \frac{e^{i2\theta} - e^{-i2\theta}}{2i} = \frac{1}{2i}\bigg{(}z^2 - \frac{1}{z^2}\bigg{)}
\end{align*}
Donc on a :
$$f(\theta) = \frac{\frac{1}{2}(z + \frac{1}{z}) \frac{1}{2i}(z^2 - \frac{1}{z^2})}{5 + \frac{3}{2} (z^2 + \frac{1}{z^2})}$$
On calcule $\widetilde{f}(z)$ :
\begin{align*}
    \widetilde{f}(z) &= \frac{1}{iz} f(\theta) \\
    &= \frac{1}{iz} \bigg{(} \frac{\frac{1}{2}(z + \frac{1}{z}) \frac{1}{2i}(z^2 - \frac{1}{z^2})}{5 + \frac{3}{2} (z^2 + \frac{1}{z^2})} \bigg{)} \\
    &= \frac{1}{iz} \bigg{(} \frac{\frac{1}{4i}(z + \frac{1}{z})(z^2 - \frac{1}{z^2})}{\frac{3}{2} (z^2 + \frac{10}{3} + \frac{1}{z^2})} \bigg{)} \\
    &= \frac{1}{iz} \frac{(z + \frac{1}{z})(z^2 - \frac{1}{z^2})}{4i} \frac{2}{3(z^2 + \frac{10}{3} + \frac{1}{z^2})} \\
    &= -\frac{(z + \frac{1}{z})(z^2 - \frac{1}{z^2})}{6z(z^2 + \frac{10}{3} + \frac{1}{z^2})} \\
    &= -\frac{z^3(z + \frac{1}{z})(z^2 - \frac{1}{z^2})}{z^36z(z^2 + \frac{10}{3} + \frac{1}{z^2})} \\
    &= -\frac{(z^2 + 1)(z^4 - 1)}{6z^2(z^4 + \frac{10}{3}z^2 + 1)} \\
    &= -\frac{(z^2 + 1)(z^4 - 1)}{6z^2(z^2+3)(z^2+\frac{1}{3})} \\
    &= \frac{-z^6-z^4+z^2+1}{6z^6+20z^4+6z^2}
\end{align*}
On détermine les pôles de $\widetilde{f}(z)$ :
\begin{align*}
    &6z^2(z^2+3)(z^2+\frac{1}{3}) = 0 \\
    \Rightarrow \quad &z_0 = 0 \text{, } z_1 = i\sqrt{\frac{1}{3}} \text{, } z_2 = -i\sqrt{\frac{1}{3}} \text{, } z_3 = i\sqrt{3} \text{, } z_4 = -i\sqrt{3}
\end{align*}
Les pôles se trouvant à l'intérieur du cercle unitaire sont $z_0 = 0$ qui est un pôle d'ordre $2$, $z_1 = i\sqrt{\frac{1}{3}}$ qui est un pôle d'ordre $1$ et $z_2 = -i\sqrt{\frac{1}{3}}$ qui est un pôle d'ordre $1$ \\
On calcule les résidus de $f$ en $z_0$, $z_1$ et $z_2$ :
\begin{align*}
    Res_0(\widetilde{f}) &= \frac{d}{dz} \bigg{(} \widetilde{f}(z)(z-0)^2 \bigg{)} \bigg{|}_{z=0} = \frac{d}{dz} \bigg{(} \frac{-z^6-z^4+z^2+1}{6z^6+20z^4+6z^2} \bigg{)} \bigg{|}_{z=0} = 0 \\
    Res_{i\sqrt{\frac{1}{3}}}(\widetilde{f}) &= \widetilde{f}(z)\bigg{(}z-i\sqrt{\frac{1}{3}}\bigg{)} \bigg{|}_{z=i\frac{1}{\sqrt{3}}} = -\frac{(z^2 + 1)(z^4 - 1)}{6z^2(z^2+3)(z+i\frac{1}{\sqrt{3}})} \bigg{|}_{z=i\frac{1}{\sqrt{3}}} = i\frac{1}{6\sqrt{3}} \\
    Res_{-i\sqrt{\frac{1}{3}}}(\widetilde{f}) &= \widetilde{f}(z)\bigg{(}z+i\sqrt{\frac{1}{3}}\bigg{)} \bigg{|}_{z=-i\frac{1}{\sqrt{3}}} = -\frac{(z^2 + 1)(z^4 - 1)}{6z^2(z^2+3)(z-i\frac{1}{\sqrt{3}})} \bigg{|}_{z=-i\frac{1}{\sqrt{3}}} = -i\frac{1}{6\sqrt{3}}
\end{align*}
Par le théorème des résidus, on obtient :
\begin{align*}
    I &= 2\pi i (Res_0 + Res_{i\sqrt{\frac{1}{3}}} + Res_{-i\sqrt{\frac{1}{3}}}) \\
    &= 2 \pi i (0 + i\frac{1}{6\sqrt{3}} -i\frac{1}{6\sqrt{3}}) \\
    &= 0
\end{align*}
