\subsection{Détermination du plus grand domaine}
Donner la partie réelle ou imaginaire d’une fonction holomorphe, utiliser les équations de Cauchy-Riemann pour trouver toutes les parties imaginaires ou réelles possibles. \\
(Série 2 exercice 3, 4 et 5, série 5 exercice 2) \\
\\
\textbf{Méthodologie :}
\begin{itemize}
    \item Séparer la partie réelle de la partie imaginaire de la fonction
    \item Établir les équations de Cauchy-Riemann et les résoudre 
    \item Intégrer $u_x(x,y)$ ou $v_x(x,y)$ en fonction de $x$, ou $u_y(x,y)$ ou $v_y(x,y)$ en fonction de $y$ pour obtenir $u(x,y)$ ou $v(x,y)$ (ne pas oublier d'ajouter la constante qui dépend de $y$ ou $x$)
    \item Déterminer la constante en dérivant le résultat précédent et en le comparant avec l'équation de Cauchy-Riemann correspondante
    \item Déterminer le plus grand domaine possible en recherchant des restrictions ou des conditions sur les variables $x$ et $y$ qui pourraient rendre la fonction indéfinie ou non holomorphe
\end{itemize}
\textbf{Exemple : }(série 2 exercice 3) \\
Déterminer le plus grand domaine de $\mathbb{C}$ où la fonction $f(z) = \log(1+z^2)$ est holomorphe. \\
\\
$z \mapsto 1+z^2$ est holomorphe sur $\mathbb{C}$ \\
$w \mapsto \log(w)$ est holomorphe sur $\mathbb{C} \setminus \{w \in \mathbb{C} \mid \Im(w) = 0, \Re(w) \leq 0\}$ \\
On pose $w = 1+z^2$ et on détermine l'ensemble des z tels que $\Im(w) = 0, \Re(w) \leq 0$ \\
$w = 1+z^2 = 1 + x^2 -y^2 + i2xy$
\begin{align*}
    \begin{cases}
        \Im(w) = 2xy = 0 \\
        \Re(w) = 1 + x^2 -y^2 \leq 0
    \end{cases}
    \quad
    \Leftrightarrow
    \quad
    \begin{cases}
        x = 0 \\
        y^2 \geq 1
    \end{cases}
\end{align*}
($y=0$ et $x^2+1 \leq 0$ est à exclure) \\
Donc le plus grand domaine est :
\begin{align*}
    D &= \{z = x+iy \mid x = 0, y^2 \geq 1\} \\
    &= \{z \in \mathbb{C} \mid x, \Re(z) = 0, |\Im(z)| \geq 1\}
\end{align*}
\textbf{Exemple :} (série 2 exercice 5) \\
Trouver une fonction holomorphe $f : \mathbb{C} \to \mathbb{C}$ dont la partie réelle est :
$$u(x,y) = e^{x^2-y^2}\cos(2xy)$$
Exprimer $f$ en fonction de $z=x+iy$ \\
On dérive $u_x(x,y)$ et $u_y(x,y)$ :
\begin{align*}
    u_x(x,y) &= 2xe^{x^2-y^2}\cos(2xy)-e^{x^2-y^2}2y\sin(2xy) \\
    &= 2e^{x^2-y^2}(x\cos(2xy) - y\sin(2xy)) \\
    u_y(x,y)& = -2ye^{x^2-y^2}\cos(2xy)-e^{x^2-y^2}2x\sin(2xy) \\
    &= -2e^{x^2-y^2}(y\cos(2xy)+x\sin(2xy))
\end{align*}
Les équations de Cauchy-Riemann donne :
\begin{align*}
    v_y(x,y) &= u_x(x,y) = 2e^{x^2-y^2}(x\cos(2xy) - y\sin(2xy)) \\
    v_x(x,y) &= -u_y(x,y) = 2e^{x^2-y^2}(y\cos(2xy)+x\sin(2xy))
\end{align*}
On intègre $v_x(x,y)$ en fonction de $x$ :
\begin{align*}
    v(x,y) &= \int v_x(x,y) dx \\
    &= \int 2e^{x^2-y^2}(y\cos(2xy)+x\sin(2xy)) dx \\
    &= e^{x^2-y^2}\sin(2xy) + \alpha(y)
\end{align*}
On dérive $v(x,y)$ en fonction de $y$ :
$$v_y(x,y) = -2ye^{x^2-y^2}\sin(2xy)+e^{x^2-y^2}\cos(2xy) + \alpha'(y)$$
On compare avec le $v_y(x,y)$ des équations de Cauchy-Riemann pour déterminer $\alpha(y)$ :
$$\Rightarrow \alpha'(y) = 0 \Rightarrow \alpha(y) = y_0 \in \mathbb{R}$$
On a donc :
$$v(x,y) = e^{x^2-y^2}\sin(2xy) + y_0$$
Et donc :
\begin{align*}
    f(z) = u(x,y) +iv(x,y) &= e^{x^2-y^2}\cos(2xy) + i(e^{x^2-y^2}\sin(2xy)+y_0) \\
    &= e^{x^2-y^2}(\cos(2xy) +i\sin(2xy)) +iy_0 \\
    &= e^{x^2-y^2}e^{i2xy} +iy_0 \\
    &= e^{x^2+i2xy-y^2} +iy_0 \\
    &= e^{z^2} + iy_0
\end{align*}