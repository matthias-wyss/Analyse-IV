\subsection{Intégration complexe}
Calculer des intégrales complexes à l’aide de la définition. \\
(Série 3 exercice 6) \\
\\
\textbf{Méthodologie :}
\begin{itemize}
    \item Choisir une paramétrisation $\gamma(t)$ de $\Gamma$ :
    $$\Gamma = \{\gamma(t) = \text{ } ? \mid t : a \to b\}$$
    \item Calculer l'intégrale avec la définition :
    $$\int_{\Gamma}fdz := \int_a^bf(\gamma(t))\gamma'(t)dt$$
\end{itemize}
\textbf{Exemple : }(série 3 exercice 6) \\
Calculer
$$I = \int_{\Gamma}(z^2+1)dz$$
où $\Gamma$ est le segment joignant $1$ à $1+i$ \\
\\
On choisit une paramétrisation $\gamma(t)$ de $\Gamma$ :
$$\Gamma = \{\gamma(t) = 1+it \mid t : 0 \to 1\}$$
On calcule l'intégrale :
\begin{align*}
    I &= \int_0^1 f(\gamma(t)) \gamma'(t) dt \\
    &= \int_0^1((1+it)^2 +1) i \text{ }dt \\
    &= \int_0^1(2 + 2it -t^2) i \text{ }dt \\
    &= \int_0^1 2i -2t - it^2 dt \\
    &= \bigg{[}2it - t^2 - \frac{it^3}{3}\bigg{]}_0^1 \\
    &= 2i - 1 - \frac{i}{3} \\
    &= -1 + \frac{5i}{3}
\end{align*}