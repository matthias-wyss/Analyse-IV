\section{Équations aux dérivées partielles (EDP)}
\subsection{Notations}
Soit $u = u(x,t) \quad x,t \in \mathbb{R}$ \\
$$u_x = \frac{\partial u}{\partial x} \quad \text{, } u_t = \frac{\partial u}{\partial t} \quad \text{, } u_{xx} = \frac{\partial^2 u}{\partial x^2}$$

\subsection{Distinctions}
\begin{itemize}
    \item Équations différentielles ordinaires (EDO) : la dérivée ne porte que sur une seule variable \\
    Exemple :
    $$u''(t)+3u(t) = sin(t)$$
    \item Équations différentielles partielles (EDP) : la dérivée porte sur plusieurs variables \\
    Exemples :
    \begin{enumerate}
        \item L'équation de la chaleur :
        $$
        \begin{cases}
            u = u(x,t) \quad \text{, }t > 0 \quad \text{, }x \in \mathbb{R} \\
            u_t = c^2u_{xx}
        \end{cases}
        $$
        \item L'équation des ondes :
        $$
        \begin{cases}
            u = u(x,t) \quad \text{, }t > 0 \quad \text{, }x \in \mathbb{R} \\
            u_{tt} = c^2u_{xx}
        \end{cases}
        $$
        \item L'équation de Laplace :
        $$
        \begin{cases}
            u = u(x,y) \quad \text{, }x,y \in \mathbb{R} \\
            u_{xx} + u_{yy} = 0
        \end{cases}
        $$
    \end{enumerate}
\end{itemize}

\subsection{Méthodes de résolution des EDP}
\begin{enumerate}
    \item Séparation des variables
    \item \begin{itemize}
        \item Avec les séries de Fourier
        \item Avec Sturm-Liouville
    \end{itemize}
\end{enumerate}

\subsection{Rappels : Transformée de Fourier}
Soit $f : \mathbb{R} \to \mathbb{R}$ , $\int_{-\infty}^{\infty} |f(x)|dx < \infty$
\begin{enumerate}
    \item La transformée de Fourier est définie par : 
    $$\mathcal{F}(f)(\alpha) = \widehat{f}(\alpha) = \frac{1}{\sqrt{2\pi}} \int_{-\infty}^{\infty} f(x) e^{-i\alpha x} dx$$
    \item Si $\int_{-\infty}^{\infty} |f(\alpha)| d\alpha < \infty$, alors la transformée inverse de Fourier est définie par :
    $$f(x) = \mathcal{F}^{-1}(\widehat{f})(x) = \frac{1}{\sqrt{2\pi}} \int_{-\infty}^{\infty} \widehat{f}(\alpha) e^{i\alpha x} d\alpha$$
    \item Transformée de Fourier de la dérivée :
    $$\mathcal{F}(f')(\alpha) = (i\alpha) \mathcal{F}(f)(\alpha) = i\alpha \widehat{f}(\alpha)$$
\end{enumerate}

\subsection{Équation de la chaleur}
\begin{itemize}
    \item Équation de la chaleur dans une barre de longueur finie : \\
    \textbf{Proposition :} Soit $c \neq 0$, $L > 0$, $f : [0;L] \to \mathbb{R}$ \\
    Le problème de l'équation de la chaleur dans une barre de longueur finie :
    $$
    \begin{cases}
        u_t(x,t) = c^2u_{xx}(x,t) \quad &\text{, }x \in ]0;L[\text{, }t \geq 0 \\
        u(0,t) = u(L,t) = 0 \quad &\text{, }t \geq 0 \\
        u(x,0) = f(x) \quad &\text{, }x \in ]0;L[
    \end{cases}
    $$
    admet comme solution :
    $$u(x,t) = \sum_{n=1}^{\infty}\beta_n \sin \bigg{(} \frac{n\pi}{L}x \bigg{)} e^{-(\frac{cn\pi}{L})^{^2} t}$$
    avec $\beta_n$ les coefficients de Fourier en sinus :
    $$\beta_n = \frac{2}{L} \int_0^Lf(x) \sin \bigg{(} \frac{n\pi}{L}x \bigg{)}$$
    \item Équation de la chaleur dans une barre de longueur infinie : \\
    \textbf{Proposition :} Soit $c \neq 0$, $f : \mathbb{R} \to \mathbb{R}$ \\
    Le problème de l'équation de la chaleur dans une barre de longueur infinie :
    $$
    \begin{cases}
        u_t(x,t) = c^2u_{xx}(x,t) \quad &\text{, }x \in \mathbb{R} \text{, }t > 0 \\
        u(x,0) = f(x) \quad &\text{, }x \in \mathbb{R}
    \end{cases}
    $$
    admet comme solution :
    $$u(x,t) = \frac{1}{\sqrt{2\pi}} \int_{-\infty}^{\infty} \widehat{f}(\alpha) e^{i\alpha x - c^2\alpha^2t} d\alpha$$
\end{itemize}

\subsection{Équation des ondes}
\textbf{Proposition :} Soit $c \in \mathbb{R}^*$, $L > 0$, $f,g : [0;L] \to \mathbb{R}$, des fonctions $\mathcal{C}^3$ telles que $f(0) = f(L) = 0$, $g(0) = g(L) = 0$\\
Le problème :
$$
\begin{cases}
    u_{tt}(x,t) = c^2u_{xx}(x,t) \quad &\text{, } x \in ]0;L[ \to \mathbb{R} \text{, } t > 0 \\
    u(0,t) = u(L,t) = 0 \quad &\text{, } t > 0 \\
    u(x,0) = f(x) \\
    u_t(x,0) = g(x)
\end{cases}
$$
admet comme solution :
$$u(x,t) = \sum_{n=1}^{\infty} \bigg{[} \alpha_n \cos \bigg{(} \frac{n\pi c}{L}t \bigg{)} + \beta_n \sin \bigg{(} \frac{n\pi c}{L}t \bigg{)} \bigg{]} \sin \bigg{(} \frac{n\pi}{L}x \bigg{)}$$
avec $\alpha_n$ et $\beta_n$ les coefficients de Fourier en sinus de $f(x)$ et $g(x)$ respectivement \\
\textbf{Proposition :} Soit $c \in \mathbb{R}^*$, $L > 0$, $f,g : [0;L] \to \mathbb{R}$, des fonctions $\mathcal{C}^3$ telles que $f(0) = f(L) = 0$, $g(0) = g(L) = 0$\\
Le problème :
$$
\begin{cases}
    u_{tt}(x,t) = c^2u_{xx}(x,t) \quad &\text{, } x \in ]0;L[ \to \mathbb{R} \text{, } t > 0 \\
    u_x(0,t) = u_x(L,t) = 0 \quad &\text{, } t > 0 \\
    u(x,0) = f(x) \\
    u_t(x,0) = g(x)
\end{cases}
$$
admet comme solution :
$$u(x,t) = \frac{\alpha_0}{2} + \frac{\beta_0}{2}t + \sum_{n=1}^{\infty} \bigg{[} \alpha_n \cos \bigg{(} \frac{n\pi c}{L}t \bigg{)} + \beta_n \sin \bigg{(} \frac{n\pi c}{L}t \bigg{)} \bigg{]} \cos \bigg{(} \frac{n\pi}{L}x \bigg{)}$$
avec $\alpha_n$ et $\beta_n$ les coefficients de Fourier en sinus de $f(x)$ et $g(x)$ respectivement